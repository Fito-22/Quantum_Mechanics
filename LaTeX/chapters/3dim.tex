\setchapterpreamble[u]{\margintoc}
\chapter{3-Dimensional Space}


We've been working with harmonic oscilators and gravitational potentials but there are other interesting potentials that we are going to mention here.

\begin{itemize}
  \item Electric Potential $V(r) = \frac{\alpha}{r}$, $\alpha>0$
  \item Confining Potential $V(r) = k r$
  \item Higgs Potential $V(r) = \frac{\alpha}{r} e^{-mr}$
\end{itemize}

\marginnote[1cm]{There are others like: Reed Potential, Cornell Potential,...}

\section{Angular Momentum in 3 Dimensions}

We can also have potentials that doesn't come from central forces. In those cases we have to look carefully at the angular momentum.

\begin{equation}
  \vec{L} = \vec{r} \times \vec{p}
\end{equation}

\begin{equation}
  \begin{array}{c}
    L_x = y p_z - z p_y\\
    L_y = z p_x - x p_z\\
    L_z = x p_y - y p_x
  \end{array}
\end{equation}

Using our knowledge from equation 2.4 we can say,

\begin{equation}
  \begin{array}{c}
    L_x = -i \hbar (y \frac{\partial}{\partial z} - z \frac{\partial}{\partial y})\\
    L_y = -i \hbar (z \frac{\partial}{\partial x} - x \frac{\partial}{\partial z})\\
    L_z = -i \hbar (x \frac{\partial}{\partial y} - y \frac{\partial}{\partial x})
  \end{array}
\end{equation}

This three components of the angular momentum and the angular momentum itself are hermitian operators. We want to see how they commute between them.

\begin{equation}
  \begin{array}{c}
  (L_xL_y) \psi = -\hbar^2 (y \frac{\partial}{\partial z} - z \frac{\partial}{\partial y})(z \frac{\partial}{\partial x} - x \frac{\partial}{\partial z}) \psi =
  \\

  \\
  = -\hbar^2 \left[y\left(\frac{\partial \psi}{\partial x}+z\frac{\partial^2\psi}{\partial x \partial z}\right)-yx\frac{\partial^2 \psi}{\partial z^2}-z^2 \frac{\partial^2 \psi}{\partial x \partial y}-zx\frac{\partial^2 \psi}{\partial y \partial z}\right] (a)
  \\

  \\
  (L_y L_x) \psi = -\hbar^2 (z \frac{\partial}{\partial x} - x \frac{\partial}{\partial z})(y \frac{\partial}{\partial z} - z \frac{\partial}{\partial y}) \psi =
  \\

  \\
  = -\hbar^2 \left[zy\frac{\partial^2\psi}{\partial x\partial z}-xy\frac{\partial^2\psi}{\partial z^2}-z^2\frac{\partial^2 \psi}{\partial x \partial y}+x\frac{\partial \psi}{\partial y}+xz\frac{\partial^2 \psi}{\partial z \partial y}\right] (b)
  \\

  \\
  (a)-(b) = [L_x,L_y] =i\hbar L_z
  \end{array}
\end{equation}

In the same way it can be prooved that:

\begin{equation}
  \begin{array}{c}
  [L_y,L_z] =i\hbar L_x
  \\

  \\
  [L_z,L_x] =i\hbar L_y
  \end{array}
\end{equation}

These 3 operators are closed under conmmutation, wich means that the conmmutation of any two of them give us the third one.

We our going to use the Levi-Civita epsilon to write the angular momentum in a more compact way.

\marginnote[1cm]{Levi-Civita epsilon has a lot of application in linear algebra. For example it can be used to determine the determinant of a matrix.}

\begin{equation}
  \begin{array}{c}
    \label{9.6}
    [L_i,L_j] = i\hbar \epsilon_{ijk} L_k
  \end{array}
\end{equation}

Now using the properties from 6.22 we want to calculate the conmmutation between $L_i$ and $L_i^2$.

\begin{equation}
  \begin{array}{c}
    [L_i,L_i^2] = [L_i,L_i]L_i + L_i[L_i,L_i] = 0
    \\

    \\
    \left[ L_i,L_j^2 \right] = \left[L_i,L_j\right]L_j + L_j[L_i,L_j] = i \hbar \left(L_k L_j \right) + i \hbar \left( L_j L_k \right) = i \hbar (L_k L_j + L_j L_k)
    \\

    \\
    \left[L_i,L_k^2\right] = [L_i,L_k]L_k + L_k[L_i,L_k] = -i\hbar\left(L_j L_k\right)-i\hbar\left(L_k L_j\right) = -i\hbar(L_k L_j + L_j L_k)
    \\

    \\
    \left[L_i,L_i^2 +L_j^2+L_k^2\right] = [L_i, L] = 0
  \end{array}
\end{equation}

This means that the angular momentum and the square of the angular momentum commute. We can simultaneously diagonalize them because they commute, so a base wher both are diagonalizable can be found.

\begin{equation}
  \label{9.8}
  \begin{array}{c}
    L^2 \ket{l,m} = \hbar^2 l^2 \ket{l,m}
    \\

    \\
    L_3 \ket{l,m} = \hbar m \ket{l,m}
  \end{array}
\end{equation}

We choose this notation for the eigenvalues to make the connection with the eigenvalues. Also we choose the notation 1,2,3 instead of x,y,z. These changes are just for convenience.

In general terms:

\begin{equation}
  \begin{array}{c}
    \left[L^2,L_c\right] = \left[\sum_{n=1}^{3}L_a^2,L_c\right] = \sum_{n=1}^{3}\left[L_a L_a ,L_c\right] =
    \\

    \\
    = \sum_{n=1}^{3} \left(L_a[L_a,L_c] + [L_a,L_c] L_a\right) =
    \\

    \\
    = i\hbar \sum_{a,b=1}^{3} \epsilon_{acb} L_a L_b + i\hbar \sum_{a,b=1}^{3} \epsilon_{acb} L_b L_a =
    \\

    \\
    i\hbar \left(\sum_{a,b=1}^{3} \epsilon_{acb} \epsilon_{bca}\right) L_a L_b = 0
  \end{array}
\end{equation}

\marginnote[-2cm]{We can use the Levi-Civita epsilon to change the order of the indices.}

Going back to equation \ref{9.8} we can say

\begin{equation}\label{9.10}
  \begin{array}{c}
    \bra{l,m}L_a^2\ket{l,m} = (\bra{l,m}L_1)(L_a\ket{l,m}) = (L_a\ket{l,m})^\dagger(L_a\ket{l,m})  \geq 0
    \\

    \\
    \bra{l,m} L_1^2 + L_2^2 \ket{l,m} \geq 0
    \\

    \\
    \bra{l,m} L_1^2 + L_2^2 + L_3^2 - L_3^2 \ket{l,m} \geq 0
    \\

    \\
    \bra{l,m} L^2\ket{l,m} - \bra{l,m} L_3^2 \ket{l,m} \geq 0
    \\

    \\
    \hbar^2 l^2 \braket{l,m}{l,m} - \hbar^2 m^2 \braket{l,m}{l,m} \geq 0
    \\

    \\
    \hbar^2 (l^2 - m^2) \geq 0
    \\

    \\
    l^2 \geq m^2
  \end{array}
\end{equation}

This result means that the proyectionof the angular momentum in the z direction is less or equal to the total angular momentum, as one would expected.

We want to create new operators $L_+$ and $L_-$ that are going to be useful to us.

\begin{equation}
  \begin{array}{c}
    L_+ = L_1 + i L_2
    \\

    \\
    L_- = L_1 - i L_2
  \end{array}
\end{equation}

With the next properties:

\begin{equation}
  \begin{array}{c}
    L_+^\dagger = L_-
    \\

    \\
    L_-^\dagger = L_+
  \end{array}
\end{equation}

As every time, when we have a new operator we want to know it commutes with the rest.

\begin{equation}
  \begin{array}{c}
    \left[L_3, L_+\right] = \left[L_3,L_1\right] + i \left[L_3,L_2\right] = i\hbar L_2 - i i\hbar L_1 = \hbar L_1 + i \hbar L_2 = \hbar L_+
    \\

    \\
    \left[L_3,L_-\right] = \left[L_3,L_1\right] - i \left[L_3,L_2\right] = i\hbar L_2 + i i\hbar L_1 = -\hbar L_1 + i \hbar L_2 = - \hbar L_-
    \\

    \\
    \left[L_+,L_-\right] = \left[L_1 +i L_2, L_1 - iL_2\right] = \left[L_1,L_1\right] + \left[L_2,L_2\right] - i\left[L_1,L_2\right] + i\left[L_2,L_1\right] = 2\hbar L_3
    \\

    \\
    \left[L^2,L_+\right] = \left[L^2,L_1\right] + i \left[L^2,L_2\right] = 0
    \\

    \\
    \left[L^2,L_-\right] = \left[L^2,L_1\right] - i \left[L^2,L_2\right] = 0
  \end{array}
\end{equation}

With this new operators we can get the next eigenvectors for m without changing l.

\begin{equation}
  \begin{array}{c}
    L^2 (L+\ket{l,m}) = L_+ L^2 \ket{l,m} = L_+ \hbar^2 l^2 \ket{l,m} = \hbar^2 l^2 (L_+ \ket{l,m})
  \end{array}
\end{equation}

This means that $L_+\ket{l,m}$ is an eigenvector of $L^2$ with the same eigenvalue $\hbar^2 l^2$.

\begin{equation}
  \begin{array}{c}
  L_3(L_+\ket{l,m}) = (\hbar L_+ + L_+ L_3)\ket{l,m} = \hbar(m+1)(L_+\ket{l,m})
  \end{array}
\end{equation}

This implies that $L_+\ket{l,m}$ is an eigenvector of $L_3$ with eigenvalue $\hbar(m+1)$, i.e, $L_+\ket{l,m} \propto \ket{l,m+1}$. We can do the same for $L_-$.

\begin{equation}
  \begin{array}{c}
    L^2(L_-\ket{l,m}) = L_-\hbar^2 l^2 \ket{l,m} = \hbar^2 l^2 (L_-\ket{l,m})
    \\

    \\
    L_3(L_-\ket{l,m}) = (-\hbar L_- + L_- L_3)\ket{l,m} = \hbar(m-1)(L_-\ket{l,m})
  \end{array}
\end{equation}

This implies that $L_-\ket{l,m}$ is an eigenvector of $L^2$ with eigenvalue $\hbar^2 l^2$ and of $L_3$ with eigenvalue $\hbar(m-1)$, i.e, $L_-\ket{l,m} \propto \ket{l,m-1}$.

We can use $L_+$ and $L_-$ to get all the eigenvalues of $L_3$, but there is a limit because for a fixed l, m can only take values from -l to l as we proved in \ref{9.10}. This can only happen if the next eigenvalue is 0.

\begin{equation}
  \begin{array}{c}
    L_+\ket{l,m_{max}} = 0
    \\

    \\
    L_-\ket{l,m_{min}} = 0
  \end{array}
\end{equation}

We are going to solve for the maximum values of m multiplying by $L_-$.

\begin{equation}
  \begin{array}{c}
    L_- L_+ \ket{l,m_{max}} = 0
    \\

    \\
    (L_1-iL_2)(L_1+iL_2)\ket{l,m_{max}} = 0
    \\

    \\
    [L_1^2+L_2^2 + i(L_1L_2-L_2L_1)]\ket{l,m_{max}} = 0
    \\

    \\
    [L^2 - L_3^2 - \hbar L_3]\ket{l,m_{max}} = 0
    \\

    \\
    (\hbar^2 l^2 - \hbar^2 m_{max}^2 - \hbar^2 m_{max})\ket{l,m_{max}} = 0
    \\

    \\
    \hbar^2 l^2 - \hbar^2 m_{max}^2 - \hbar^2 m_{max} = 0
    \\

    \\
    l^2 = m_{max}^2 + m_{max}
  \end{array}
\end{equation}

Which implies than the maximum m is less than l, this can only happen because $L^2$ and $L_3$ are operators. We can do the same for the minimum value of m.

\begin{equation}
  \begin{array}{c}
    L_+ L_- \ket{l,m_min} = 0
    \\

    \\
    (L_1+iL_2)(L_1-iL_2)\ket{l,m_min} = 0
    \\

    \\
    [L_1^2+L_2^2 - i(L_1L_2-L_2L_1)]\ket{l,m_min} = 0
    \\

    \\
    [L^2 - L_3^2 + \hbar L_3]\ket{l,m_min} = 0
    \\

    \\
    (\hbar^2 l^2 - \hbar^2 m_{min}^2 + \hbar^2 m_{min})\ket{l,m_{min}} = 0
    \\

    \\
    \hbar^2 l^2 - \hbar^2 m_{min}^2 + \hbar^2 m_{min} = 0
    \\

    \\
    l^2 = m_{min}^2 - m_{min}
  \end{array}
\end{equation}

If we compare the limits of m we end up with the next result.

\begin{equation}
  \begin{array}{c}
    m^2_{max} + m_{max} = m^2_{min} - m_{min}
  \end{array}
\end{equation}

This expresion have two solutions:

\begin{equation}
  \begin{array}{c}
    m_{max} = m_{min} - 1
    \\

    \\
    m_{max} = - m_{min}
  \end{array}
\end{equation}

The first solution can be discarted because it would imply that $m_{max}$ is less than $m_{min}$. The second solution is the real one. We know that m increases or decreases by a factor of 1, so we can say that:

\begin{equation}
  \begin{array}{c}
    m_{max} = m_{min} + I
  \end{array}
\end{equation}

Where I is an non-negative integer. If we used what we know about the limits of m we can say that:

\begin{equation}
  \begin{array}{c}
    m_{max} = - m_{max} + I
    \\

    \\
    m_{max} = \frac{I}{2}
  \end{array}
\end{equation}

This means that $L_z$ is quantized, i. e., only some proyections of the angular momentum in the z axis are allowed. We are going to start labeling the eigenfunctions as $\ket{j,m}$ where j is defined by:

\begin{equation}
  \begin{array}{c}
    j = m_{max}
    \\

    \\
    l^2 = j(j+1)
  \end{array}
\end{equation}

Where j can be the natural numbers or half of them, i.e., j = 0, 1/2, 1, 3/2... Now our eigenfunctions follow:

\begin{equation}
  \begin{array}{c}
    L^2 \ket{j,m} = \hbar^2 j(j+1) \ket{j,m}
    \\

    \\
    L_3 \ket{j,m} = \hbar m \ket{j,m}
  \end{array}
\end{equation}

For m = j, j-1, j-2,..., -j. The j,m pair make a completeset of (orthogonal) eigenvectors. Following what we know from \ref{6.29}:

\begin{equation}
  \begin{array}{c}
    \braket{j_1,m_1}{j_1,m_2} = 0 \text{ if } m_1 \neq m_2
    \\

    \\
    \braket{j_1,m_1}{j_2,m_1} = 0 \text{ if } j_1 \neq j_2
  \end{array}
\end{equation}

We also set the normalization condition:

\begin{equation}
  \begin{array}{c}
    \braket{j,m}{j,m} = 1
  \end{array}
\end{equation}

From now we will move on to what an experimentalist will measure.

\section{Angular Momentum as a vector operator}

We can interpret the angular momentum as:

\begin{equation}
  \begin{array}{c}
    \vec{L} = L_1 \hat{i} + L_2 \hat{j} + L_3 \hat{k}
  \end{array}
\end{equation}

In a classical way we know that the angular mometum is conserved, so we can say that all the angular momentum is in one direction.

\begin{equation}
  \begin{array}{c}
    \vec{L} x \vec{L} = (L_2 L_3 - L_3L_2)\hat{i} + (L_3L_1 - L_1L_3)\hat{j} + (L_1L_2 - L_2L_1)\hat{k}
  \end{array}
\end{equation}

In classical mechanics that product is 0 because the components are numbers but in a quantum way they are operators that doesn't commute, in this case the cross product is:

\begin{equation}
  \begin{array}{c}
    \vec{L} x \vec{L} = i\hbar \vec{L}
  \end{array}
\end{equation}

This is the operator equation. On an experiment we can measure the angular momentum of the electron in hydrogen. Assume it is in one state $\psi$.

\begin{equation}
  \begin{array}{c}
    \int_{-\infty}^{\infty} dxdydz \psi^{\star}(x,y,z) O \psi(x,y,z) = \text{ Measurement of O}
  \end{array}
\end{equation}

We are going to make a distintion between the angular momentum $\vec{L}$, and the measured angular momentum, $\vec{l}$.

\begin{equation}
  \begin{array}{c}
    \vec{l} = \bra{s}\vec{L}\ket{s}
    \\

    \\
    \vec{l} = \bra{s}L_1\ket{s}\hat{i} + \bra{s}L_2\ket{s}\hat{j} + \bra{s}L_3\ket{s}\hat{k}
  \end{array}
\end{equation}

With the next properties:

\begin{equation}
  \begin{array}{c}
    \vec{l} x \vec{l} = 0
    \\

    \\
    \vec{l} \cdot \vec{l} = (\bra{s}L_1\ket{s})^2 + (\bra{s}L_2\ket{s})^2 + (\bra{s}L_3\ket{s})^2 \geq 0
  \end{array}
\end{equation}

We want to know the value of this dot product.

\begin{equation}
  \begin{array}{c}
    \bra{j,m}L_3\ket{j,m} = \hbar m \braket{j,m} = \hbar m
    \\

    \\
    \bra{j,m}L_3^2\ket{j,m} = \hbar^2 m^2 = (\bra{j,m}L_3\ket{j,m})^2
    \\

    \\
    \bra{j,m}L^2\ket{j,m} = \hbar^2 j(j+1)
    \\

    \\
    \bra{j,m}L_1^2 + L_2^2\ket{j,m} = \hbar^2 ( j(j+1) - m^2 )
  \end{array}
\end{equation}

To get the value of $\bra{j,m}L_1\ket{j,m}$ we have to use the definition of $L_+$ and $L_-$.

\begin{equation}
  \begin{array}{c}
    \bra{j,m}L_1\ket{j,m} = \bra{j,m}L_+\ket{j,m} + \bra{j,m}iL_-\ket{j,m} =
    \\

    \\
    = N_+ \braket{j,m}{j,m+1} + N_- \braket{j,m}{j,m-1} = 0
  \end{array}
\end{equation}

The same argument can be done for $L_2$. Because of this we can say that:

\begin{equation}\left\{
  \begin{array}{c}
    l_1 = 0
    \\

    \\
    l_2 = 0
    \\

    \\
    l_3 = \hbar m
    \\

    \\
    l^2 = ?
  \end{array}\right.
\end{equation}

We have two definitions if we consider $l^2$ as the measure of $L^2$ is going to be $l^2$ = $\hbar^2 j(j+1)$, but if we consider $l^2$ as the square of the measure L we get $\hbar^2 m^2$. This is exactly the same as talking about the square of the mean or taking about the variance, so we are familiarize with this concepts from chapter 2.

We said before that the operators $L_+$ and $L_-$ are proportional to the next or the previous eigenfunctions. Now we want to find the exact value of the proportionality constant.

\begin{equation}
  \begin{array}{c}
    \L_+ \ket{j,m} = n_{+,j,m} \ket{j,m+1}
    \\

    \\
    L_- \ket{j,m} = n_{-,j,m} \ket{j,m-1}
    \\

    \\
    \braket{j_1,m_1}{j_2,m_2} = \delta_{j_1,j_2}\delta_{m_1,m_2}
  \end{array}
\end{equation}

We can always find real positive values for both $n_+$ and $n_-$. We want them to be like this because we are going to take the conjugate of the expresion above.

\begin{equation}
  \begin{array}{c}
    \bra{j,m}L_- = n_{+,j,m} \bra{j,m+1}
    \\

    \\
    (\bra{j,m}L_-)(L_+\ket{j,m}) = n_{+,j,m}^2  \braket{j,m+1}{j,m+1}
    \\

    \\
    \bra{j,m}(L_1-iL_2)(L_1+iL_2)\ket{j,m} = n_{+,j,m}^2
    \\

    \\
    \bra{j,m}L_1^2+L_2^2-\hbar L_3\ket{j,m} = n_{+,j,m}^2
    \\

    \\
    \hbar^2 j(j+1) - \hbar^2 m^2 - \hbar^2 m \braket{j,m}{j,m} = n_{+,j,m}^2
    \\

    \\
    \hbar^2 [j(j+1) - m(m+1)] = n_{+,j,m}^2
  \end{array}
\end{equation}

We have finally reach the value of $n_{+,j,m}$. We can write now:

\begin{equation}
  \begin{array}{c}
    L_+ \ket{j,m} = \hbar \sqrt{j(j+1) - m(m+1)} \ket{j,m+1}
  \end{array}
\end{equation}

It is easy to prove from this that $L_+\ket{j,j} = 0$, as we wanted. We can do the same for $L_-$.

\begin{equation}
  \begin{array}{c}
    \bra{j,m}L_+ = n_{-,j,m} \bra{j,m-1}
    \\

    \\
    (\bra{j,m}L_+)(L_-\ket{j,m}) = n_{-,j,m}^2  \braket{j,m-1}{j,m-1}
    \\

    \\
    \bra{j,m}(L_1+iL_2)(L_1-iL_2)\ket{j,m} = n_{-,j,m}^2
    \\

    \\
    \bra{j,m}L_1^2+L_2^2+\hbar L_3\ket{j,m} = n_{-,j,m}^2
    \\

    \\
    \hbar^2 j(j+1) - \hbar^2 m^2 + \hbar^2 m \braket{j,m}{j,m} = n_{-,j,m}^2
    \\

    \\
    \hbar^2 [j(j+1) - m(m-1)] = n_{-,j,m}^2
  \end{array}
\end{equation}

We can write now:

\begin{equation}
  \begin{array}{c}
    L_- \ket{j,m} = \hbar \sqrt{j(j+1) - m(m-1)} \ket{j,m+1}
  \end{array}
\end{equation}

And for $m=-j$ we get 0 again, as we wanted. Knowing $L_+$ and $L_$ we can get the values for all the three components of the angular momentum.

\begin{equation}
  \begin{array}{c}
    L_1 = \frac{L_+ + L_-}{2}
    \\

    \\
    L_2 = \frac{L_+ - L_-}{2i}
  \end{array}
\end{equation}

The components of $\vec{L}$ acting on $\ket{j,m}$ are:



\begin{equation*}
  \label{9.43}
  \begin{array}{l}
    L_1 \ket{j,m} = \frac{\hbar}{2}(\sqrt{j(j+1) - m(m+1)}\ket{j,m+1} + \sqrt{j(j+1) - m(m-1)}\ket{j,m-1})
    \\

    \\
    L_2 \ket{j,m} = \frac{\hbar i}{2}(\sqrt{j(j+1) - m(m+1)}\ket{j,m+1} - \sqrt{j(j+1) - m(m-1)}\ket{j,m-1})
    \\

    \\
    L_3 \ket{j,m} = \hbar m \ket{j,m}
  \end{array}
\end{equation*}

Using the knowledge from chapter 6, if we want to get the excat values for every component of the three operators we can apply \ref{6.16}.


\begin{equation*}
  \begin{array}{l}
    (L_1)_{j_1m_1,j_2,m_2} = \bra{j_1,m_1}L_1\ket{j_2,m_2} = \frac{\hbar}{2}[\sqrt{j_2(j_2+1)-m_2(m_2+1)}\delta_{j_1j_2,m_1m_2+1} + \sqrt{j_2(j_2+1)-m_2(m_2-1)}\delta_{j_1j_2,m_1m_2-1}]
    \\

    \\
    (L_2)_{j_1m_1,j_2,m_2} = \bra{j_1,m_1}L_2\ket{j_2,m_2} = \frac{-\hbar i}{2}[\sqrt{j_2(j_2+1)-m_2(m_2+1)}\delta_{j_1j_2,m_1m_2+1} - \sqrt{j_2(j_2+1)-m_2(m_2-1)}\delta_{j_1j_2,m_1m_2-1}]
    \\

    \\
    (L_3)_{j_1m_1,j_2,m_2} = \bra{j_1,m_1}L_3\ket{j_2,m_2} = \hbar m_2 \delta_{j_1j_2,m_1m_2}
  \end{array}
\end{equation*}

We can see at first sight that $L_3$ has to be a diagonal matrix, while $L_1$ and $L_2$ can only have non-zero values on the secondary diagonals. In the next section we will get the values of the matrix for some j.

\pagelayout{wide} % No margins

\section{Matrix representation for j={1, 1/2, 3/2, 2}}


The case for j=0 is a trivial case. We are going to start with j=1, where the set of values for m is m={1,0,-1}. We are going to define the matrices as:

\begin{equation}
  L_a =
    \left[\begin{matrix}
      (L_a)_{-1,-1} & (L_a)_{-1,0} & (L_a)_{-1,1}\\
      (L_a)_{0,-1} & (L_a)_{0,0} & (L_a)_{0,1}\\
      (L_a)_{1,-1} & (L_a)_{1,0} & (L_a)_{1,1}
    \end{matrix}\right]
\end{equation}

Where the subindixes of the components represent $m_1$ and $m_2$. We can use the expresions from the last equation fo the previous chapter to get the matrix representation of the angular momentum.


\begin{equation}
  L_1 = \frac{\hbar}{\sqrt{2}}
    \left[\begin{matrix}
      0 & 1 & 0\\
      1 & 0 & 1\\
      0 & 1 & 0
    \end{matrix}\right],
  L_2 = \frac{i\hbar}{\sqrt{2}}
    \left[\begin{matrix}
      0 & 1 & 0\\
      -1 & 0 & 1\\
      0 & -1 & 0
    \end{matrix}\right],
  L_3 = \hbar
    \left[\begin{matrix}
      -1 & 0 & 0\\
      0 & 0 & 0\\
      0 & 0 & 1
    \end{matrix}\right]
\end{equation}

We can find the other operators we are interested in just aplying matrix multiplication.

\begin{equation}
  \begin{array}{c}
  L_1^2 = \frac{\hbar}{\sqrt{2}}
  \left[\begin{matrix}
    0 & 1 & 0\\
      1 & 0 & 1\\
      0 & 1 & 0
  \end{matrix}
  \right]\frac{\hbar}{\sqrt{2}}
  \left[\begin{matrix}
    0 & 1 & 0\\
      1 & 0 & 1\\
      0 & 1 & 0
  \end{matrix}
  \right] = \frac{\hbar^2}{2}
  \left[\begin{matrix}
    1 & 0 & 1\\
      0 & 2 & 0\\
      1 & 0 & 1
  \end{matrix}
  \right]
  \\

  \\
  L_2^2 = \frac{i\hbar}{\sqrt{2}}
  \left[\begin{matrix}
    0 & 1 & 0\\
    -1 & 0 & 1\\
    0 & -1 & 0
  \end{matrix}\right]
  \frac{i\hbar}{\sqrt{2}}
    \left[\begin{matrix}
      0 & 1 & 0\\
      -1 & 0 & 1\\
      0 & -1 & 0
    \end{matrix}\right] =
    \frac{\hbar^2}{2}
    \left[\begin{matrix}
      1 & 0 & -1\\
      0 & 2 & 0\\
      -1 & 0 & 1
    \end{matrix}\right]
    \\

    \\
    L_3^2 = \hbar
    \left[\begin{matrix}
      -1 & 0 & 0\\
      0 & 0 & 0\\
      0 & 0 & 1
    \end{matrix}\right]
    \hbar
    \left[\begin{matrix}
      -1 & 0 & 0\\
      0 & 0 & 0\\
      0 & 0 & 1
    \end{matrix}\right] =
    \hbar^2
    \left[\begin{matrix}
      1 & 0 & 0\\
      0 & 0 & 0\\
      0 & 0 & 1
    \end{matrix}\right]
    \\

    \\
    L^2 = L_1^2 + L_2^2 + L_3^2 = 2\hbar^2
    \left[\begin{matrix}
      1 & 0 & 0\\
      0 & 1 & 0\\
      0 & 0 & 1
    \end{matrix}\right]
  \end{array}
\end{equation}

We can see that $L^2$ is also a diagonal amtrix as we expected it to be, this means that we did not make an algebra mistake.

Just to make sure that we are doing everything right, we can check that the conmmutation properties are satisfied.

\begin{equation}
  \begin{array}{c}
    [L_1,L_2] = L_1L_2-L_2L_1 =
    \frac{\hbar}{\sqrt{2}}
    \left[\begin{matrix}
      0 & 1 & 0\\
      1 & 0 & 1\\
      0 & 1 & 0
    \end{matrix}\right]
    \frac{i\hbar}{\sqrt{2}}
    \left[\begin{matrix}
      0 & 1 & 0\\
      -1 & 0 & 1\\
      0 & -1 & 0
    \end{matrix}\right] -
    \frac{i\hbar}{2}
    \left[\begin{matrix}
      0 & 1 & 0\\
      -1 & 0 & 1\\
      0 & -1 & 0
    \end{matrix}\right]
    \frac{\hbar}{\sqrt{2}}
    \left[\begin{matrix}
      0 & 1 & 0\\
      1 & 0 & 1\\
      0 & 1 & 0
    \end{matrix}\right] =
    \\

    \\
    =
    \frac{i\hbar^2}{2}
    \left[\begin{matrix}
      -1 & 0 & 1\\
      0 & 0 & 0\\
      -1 & 0 & 1
    \end{matrix}\right]-
    \frac{i\hbar^2}{2}
    \left[
    \begin{matrix}
      1 & 0 & 1\\
      0 & 0 & 0\\
      -1 & 0 & -1
    \end{matrix}
    \right] = i \hbar \hbar\left[\begin{matrix}
      -1 & 0 & 0\\
      0 & 0 & 0\\
      0 & 0 & 1
    \end{matrix}\right] = i \hbar L_3
  \end{array}
\end{equation}

We can do the same for the other commutators.

Now we will work the math or j=1/2. The operator are gonna look like:

\begin{equation}
  L_a =
    \left[\begin{matrix}
      (L_a)_{-1/2,-1/2} & (L_a)_{-1/2,1/2}\\
      (L_a)_{1/2,-1/2} & (L_a)_{1/2,1/2}
    \end{matrix}\right]
\end{equation}

And in particular the operators are:

\begin{equation}
  L_1 = \frac{\hbar}{2}
    \left[\begin{matrix}
      0 & 1\\
      1 & 0
    \end{matrix}\right],
  L_2 = \frac{i\hbar}{2}
    \left[\begin{matrix}
      0 & 1\\
      -1 & 0
    \end{matrix}\right],
  L_3 = \frac{\hbar}{2}
    \left[\begin{matrix}
      -1 & 0\\
      0 & 1
    \end{matrix}\right]
\end{equation}

We can find the other operators we are interested in just aplying matrix multiplication.

\begin{equation}
  \begin{array}{c}
  L_1^2 = \frac{\hbar^2}{4}
  \left[
  \begin{matrix}
    0 & 1\\
    1 & 0
  \end{matrix}\right]
  \left[
    \begin{matrix}
      0 & 1\\
      1 & 0
    \end{matrix}\right] =
  \frac{\hbar^2}{4} \left[\begin{matrix}
  1 & 0\\
  0 & 1
  \end{matrix}
  \right]
  \\

  \\
  L_2^2 = \frac{-\hbar^2}{4} \left[
    \begin{matrix}
      0 & 1\\
      -1 & 0
    \end{matrix}\right]\left[\begin{matrix}
      0 & 1\\
      -1 & 0
    \end{matrix}
  \right] =
  \frac{\hbar^2}{4} \left[
    \begin{matrix}
      1 & 0\\
      0 & 1
    \end{matrix}
  \right]
  \\

  \\
  L_3^2 = \frac{\hbar^2}{4}
  \left[\begin{matrix}
    -1 & 0\\
    0 & 1
  \end{matrix}\right]\left[\begin{matrix}
    -1 & 0\\
    0 & 1
  \end{matrix}\right] =
  \frac{\hbar^2}{4}
  \left[\begin{matrix}
    1 & 0\\
    0 & 1
  \end{matrix}\right]
  \\

  \\
  L^2 = L_1^2 + L_2^2 + L_3^2 = \frac{3\hbar^2}{4}\left[\begin{matrix}
    1 & 0\\
    0 & 1
  \end{matrix}\right]
  \end{array}
\end{equation}

We know we did the math correct because $L^2$ is a diagonal matrix with values $\hbar^2j(j+1)$.

\begin{equation}
  \begin{array}{c}
    [L_1,L_2] = L_1L_2-L_2L_1 = \frac{\hbar}{2}\left[
      \begin{matrix}
        0 & 1\\
        1 & 0
      \end{matrix}\right]\frac{i\hbar}{2}
      \left[\begin{matrix}
        0 & 1\\
        -1 & 0
      \end{matrix}\right] - \frac{i\hbar}{2}\left[\begin{matrix}
        0 & 1\\
        -1 & 0
      \end{matrix}\right]\frac{\hbar}{2}\left[
        \begin{matrix}
          0 & 1\\
          1 & 0
        \end{matrix}\right] =
      \\

      \\
      = i \hbar \frac{\hbar}{2}\left[\begin{matrix}
        -1 & 0\\
        0 & 1
      \end{matrix}\right] = i \hbar L_3
  \end{array}
\end{equation}

The conmmutation properties are satisfied. We can do the same for j=3/2. To begin with our matrices are going to look like:

\begin{equation}
  L_a =
    \left[\begin{matrix}
      (L_a)_{-3/2,-3/2} & (L_a)_{-3/2,-1/2} & (L_a)_{-3/2,1/2} & (L_a)_{-3/2,3/2}\\
      (L_a)_{-1/2,-3/2} & (L_a)_{-1/2,-1/2} & (L_a)_{-1/2,1/2} & (L_a)_{-1/2,3/2}\\
      (L_a)_{1/2,-3/2} & (L_a)_{1/2,-1/2} & (L_a)_{1/2,1/2} & (L_a)_{1/2,3/2}\\
      (L_a)_{3/2,-3/2} & (L_a)_{3/2,-1/2} & (L_a)_{3/2,1/2} & (L_a)_{3/2,3/2}
    \end{matrix}\right]
\end{equation}

The components of the angular momentum are:

\begin{equation}
  L_1 = \frac{\hbar}{2}\left[\begin{matrix}
    0 & \sqrt{3} & 0 & 0\\
    \sqrt{3} & 0 & 2 & 0\\
    0 & 2 & 0 & \sqrt{3}\\
    0 & 0 & \sqrt{3} & 0
  \end{matrix}\right],
  L_2 = \frac{i\hbar}{2}\left[\begin{matrix}
    0 & \sqrt{3} & 0 & 0\\
    -\sqrt{3} & 0 & 2 & 0\\
    0 & -2 & 0 & \sqrt{3}\\
    0 & 0 & -\sqrt{3} & 0
  \end{matrix}\right],
  L_3 = \frac{\hbar}{2}\left[\begin{matrix}
    -3 & 0 & 0 & 0\\
    0 & -1 & 0 & 0\\
    0 & 0 & 1 & 0\\
    0 & 0 & 0 & 3
  \end{matrix}\right]
\end{equation}

We can find the other operators we are interested in aplying matrix multiplication.

\begin{equation}
  \begin{array}{c}
    L_1^2 = \frac{\hbar^2}{4}\left[\begin{matrix}
      0 & \sqrt{3} & 0 & 0\\
      \sqrt{3} & 0 & 2 & 0\\
      0 & 2 & 0 & \sqrt{3}\\
      0 & 0 & \sqrt{3} & 0
    \end{matrix}\right]\left[\begin{matrix}
      0 & \sqrt{3} & 0 & 0\\
      \sqrt{3} & 0 & 2 & 0\\
      0 & 2 & 0 & \sqrt{3}\\
      0 & 0 & \sqrt{3} & 0
    \end{matrix}\right] =
    \frac{\hbar^2}{4}\left[\begin{matrix}
      3 & 0 & 2\sqrt{3} & 0\\
      0 & 7 & 0 & 2\sqrt{3}\\
      2\sqrt{3} & 0 & 7 & 0\\
      0 & 2\sqrt{3} & 0 & 3
    \end{matrix}\right]
    \\

    \\
    L_2^2 = \frac{-\hbar^2}{4}\left[\begin{matrix}
      0 & \sqrt{3} & 0 & 0\\
      -\sqrt{3} & 0 & 2 & 0\\
      0 & -2 & 0 & \sqrt{3}\\
      0 & 0 & -\sqrt{3} & 0
    \end{matrix}\right]\left[\begin{matrix}
      0 & \sqrt{3} & 0 & 0\\
      -\sqrt{3} & 0 & 2 & 0\\
      0 & -2 & 0 & \sqrt{3}\\
      0 & 0 & -\sqrt{3} & 0
    \end{matrix}\right] =
    \frac{-\hbar^2}{4}\left[\begin{matrix}
      -3 & 0 & 2\sqrt{3} & 0\\
      0 & -7 & 0 & 2\sqrt{3}\\
      2\sqrt{3} & 0 & -7 & 0\\
      0 & 2\sqrt{3} & 0 & -3
    \end{matrix}\right]
    \\

    \\
    L_3^2 = \frac{\hbar^2}{4}\left[\begin{matrix}
      -3 & 0 & 0 & 0\\
      0 & -1 & 0 & 0\\
      0 & 0 & 1 & 0\\
      0 & 0 & 0 & 3
    \end{matrix}\right]\left[\begin{matrix}
      -3 & 0 & 0 & 0\\
      0 & -1 & 0 & 0\\
      0 & 0 & 1 & 0\\
      0 & 0 & 0 & 3
    \end{matrix}\right] =
    \frac{\hbar^2}{4}\left[\begin{matrix}
      9 & 0 & 0 & 0\\
      0 & 1 & 0 & 0\\
      0 & 0 & 1 & 0\\
      0 & 0 & 0 & 9
    \end{matrix}\right]
    \\

    \\
    L^2 = L_1^2 + L_2^2 + L_3^2 = \frac{\hbar^2}{4} \left[\begin{matrix}
      15 & 0 & 0 & 0\\
      0 & 15 & 0 & 0\\
      0 & 0 & 15 & 0\\
      0 & 0 & 0 & 15
    \end{matrix}\right] = \frac{15\hbar^2}{4} I_{4x4}
  \end{array}
\end{equation}

We did a correct calculation because $L^2$ is a diagonal matrix with values $\hbar^2j(j+1)$. Let's Prove the conmmutation properties.

\begin{equation}
  \begin{array}{c}
    [L_1,L_2] = L_1L_2-L_2L_1 =
    \\

    \\
    \frac{\hbar}{2}\left[\begin{matrix}
      0 & \sqrt{3} & 0 & 0\\
      \sqrt{3} & 0 & 2 & 0\\
      0 & 2 & 0 & \sqrt{3}\\
      0 & 0 & \sqrt{3} & 0
    \end{matrix}\right]\frac{i\hbar}{2}\left[\begin{matrix}
      0 & \sqrt{3} & 0 & 0\\
      -\sqrt{3} & 0 & 2 & 0\\
      0 & -2 & 0 & \sqrt{3}\\
      0 & 0 & -\sqrt{3} & 0
    \end{matrix}\right] -
    \frac{i\hbar}{2}\left[\begin{matrix}
      0 & \sqrt{3} & 0 & 0\\
      -\sqrt{3} & 0 & 2 & 0\\
      0 & -2 & 0 & \sqrt{3}\\
      0 & 0 & -\sqrt{3} & 0
    \end{matrix}\right]\frac{\hbar}{2}\left[\begin{matrix}
      0 & \sqrt{3} & 0 & 0\\
      \sqrt{3} & 0 & 2 & 0\\
      0 & 2 & 0 & \sqrt{3}\\
      0 & 0 & \sqrt{3} & 0
    \end{matrix}\right] =
    \\

    \\
    = \frac{i\hbar^2}{4}\left[\begin{matrix}
      -3 & 0 & 2\sqrt{3} & 0\\
      0 & -1 & 0 & 2\sqrt{3}\\
      -2\sqrt{3} & 0 & 1 & 0\\
      0 & -2\sqrt{3} & 0 & 3
    \end{matrix}\right] - \frac{i\hbar^2}{4}\left[\begin{matrix}
      3 & 0 & 2\sqrt{3} & 0\\
      0 & 1 & 0 & 2\sqrt{3}\\
      -2\sqrt{3} & 0 & -1 & 0\\
      0 & -2\sqrt{3} & 0 & -3
    \end{matrix}\right] =
    \\

    \\
    = i\hbar \frac{\hbar}{2}\left[\begin{matrix}
      -3 & 0 & 0 & 0\\
      0 & -1 & 0 & 0\\
      0 & 0 & 1 & 0\\
      0 & 0 & 0 & 3
    \end{matrix}\right] = i \hbar L_3
  \end{array}
\end{equation}

Everything seems in order, let's move to the last one, j=2. The matrices are gonna be defined by

\begin{equation}
  L_a =
    \left[\begin{matrix}
      (L_a)_{-2,-2} & (L_a)_{-2,-1} & (L_a)_{-2,0} & (L_a)_{-2,1} & (L_a)_{-2,2}\\
      (L_a)_{-1,-2} & (L_a)_{-1,-1} & (L_a)_{-1,0} & (L_a)_{-1,1} & (L_a)_{-1,2}\\
      (L_a)_{0,-2} & (L_a)_{0,-1} & (L_a)_{0,0} & (L_a)_{0,1} & (L_a)_{0,2}\\
      (L_a)_{1,-2} & (L_a)_{1,-1} & (L_a)_{1,0} & (L_a)_{1,1} & (L_a)_{1,2}\\
      (L_a)_{2,-2} & (L_a)_{2,-1} & (L_a)_{2,0} & (L_a)_{2,1} & (L_a)_{2,2}
    \end{matrix}\right]
\end{equation}

The components of the angular momentum are:

\begin{equation}
  \begin{array}{c}
    L_1 = \frac{\hbar}{2}\left[\begin{matrix}
      0 & 2 & 0 & 0 & 0\\
      2 & 0 & \sqrt{6} & 0 & 0\\
      0 & \sqrt{6} & 0 & \sqrt{6} & 0\\
      0 & 0 & \sqrt{6} & 0 & 2\\
      0 & 0 & 0 & 2 & 0
    \end{matrix}\right],
    L_2 = \frac{i\hbar}{2}\left[\begin{matrix}
      0 & 2 & 0 & 0 & 0\\
      -2 & 0 & \sqrt{6} & 0 & 0\\
      0 & -\sqrt{6} & 0 & \sqrt{6} & 0\\
      0 & 0 & -\sqrt{6} & 0 & 2\\
      0 & 0 & 0 & -2 & 0
    \end{matrix}\right],
    \\

    \\
    L_3 = \hbar\left[\begin{matrix}
      -2 & 0 & 0 & 0 & 0\\
      0 & -1 & 0 & 0 & 0\\
      0 & 0 & 0 & 0 & 0\\
      0 & 0 & 0 & 1 & 0\\
      0 & 0 & 0 & 0 & 2
    \end{matrix}\right]
  \end{array}
\end{equation}

The other operators are:

\begin{equation}
  \begin{array}{c}
    L_1^2 =\frac{\hbar^2}{4}\left[\begin{matrix}
      0 & 2 & 0 & 0 & 0\\
      2 & 0 & \sqrt{6} & 0 & 0\\
      0 & \sqrt{6} & 0 & \sqrt{6} & 0\\
      0 & 0 & \sqrt{6} & 0 & 2\\
      0 & 0 & 0 & 2 & 0
    \end{matrix}\right]\left[\begin{matrix}
      0 & 2 & 0 & 0 & 0\\
      2 & 0 & \sqrt{6} & 0 & 0\\
      0 & \sqrt{6} & 0 & \sqrt{6} & 0\\
      0 & 0 & \sqrt{6} & 0 & 2\\
      0 & 0 & 0 & 2 & 0
    \end{matrix}\right] =
    \frac{\hbar^2}{4}\left[\begin{matrix}
      4 & 0 & 2\sqrt{6} & 0 & 0\\
      0 & 10 & 0 & 6 & 0\\
      2\sqrt{6} & 0 & 12 & 0 & 2\sqrt{6}\\
      0 & 6 & 0 & 10 & 0\\
      0 & 0 & 2\sqrt{6} & 0 & 4
    \end{matrix}\right]
    \\

    \\
    L_2^2 = \frac{-\hbar^2}{4}\left[\begin{matrix}
      0 & 2 & 0 & 0 & 0\\
      -2 & 0 & \sqrt{6} & 0 & 0\\
      0 & -\sqrt{6} & 0 & \sqrt{6} & 0\\
      0 & 0 & -\sqrt{6} & 0 & 2\\
      0 & 0 & 0 & -2 & 0
    \end{matrix}\right]\left[\begin{matrix}
      0 & 2 & 0 & 0 & 0\\
      -2 & 0 & \sqrt{6} & 0 & 0\\
      0 & -\sqrt{6} & 0 & \sqrt{6} & 0\\
      0 & 0 & -\sqrt{6} & 0 & 2\\
      0 & 0 & 0 & -2 & 0
    \end{matrix}\right] =
    \frac{-\hbar^2}{4}\left[\begin{matrix}
      -4 & 0 & 2\sqrt{6} & 0 & 0\\
      0 & -10 & 0 & 6 & 0\\
      2\sqrt{6} & 0 & -12 & 0 & 2\sqrt{6}\\
      0 & 6 & 0 & -10 & 0\\
      0 & 0 & 2\sqrt{6} & 0 & -4
    \end{matrix}\right]
    \\

    \\
    L_3^2 = \hbar^2\left[\begin{matrix}
      -2 & 0 & 0 & 0 & 0\\
      0 & -1 & 0 & 0 & 0\\
      0 & 0 & 0 & 0 & 0\\
      0 & 0 & 0 & 1 & 0\\
      0 & 0 & 0 & 0 & 2
    \end{matrix}\right]\left[\begin{matrix}
      -2 & 0 & 0 & 0 & 0\\
      0 & -1 & 0 & 0 & 0\\
      0 & 0 & 0 & 0 & 0\\
      0 & 0 & 0 & 1 & 0\\
      0 & 0 & 0 & 0 & 2
    \end{matrix}\right] =
    \hbar^2\left[\begin{matrix}
      4 & 0 & 0 & 0 & 0\\
      0 & 1 & 0 & 0 & 0\\
      0 & 0 & 0 & 0 & 0\\
      0 & 0 & 0 & 1 & 0\\
      0 & 0 & 0 & 0 & 4
    \end{matrix}\right]
    \\

    \\
    L^2 = L_1^2+L_2^2+L_3^2 = \hbar^2 \left[\begin{matrix}
      6 & 0 & 0 & 0 & 0\\
      0 & 6 & 0 & 0 & 0\\
      0 & 0 & 6 & 0 & 0\\
      0 & 0 & 0 & 6 & 0\\
      0 & 0 & 0 & 0 & 6
    \end{matrix}\right]
  \end{array}
\end{equation}

Again the math is correct because $L^2$ is a diagonal matrix with values $\hbar^2j(j+1)$. Let's prove the conmmutation properties.

\begin{equation}
  \begin{array}{c}
    [L_1,L_2] = L_1L_2-L_2L_1 =
    \\

    \\
    = \frac{\hbar}{2}\left[\begin{matrix}
      0 & 2 & 0 & 0 & 0\\
      2 & 0 & \sqrt{6} & 0 & 0\\
      0 & \sqrt{6} & 0 & \sqrt{6} & 0\\
      0 & 0 & \sqrt{6} & 0 & 2\\
      0 & 0 & 0 & 2 & 0
    \end{matrix}\right]\frac{i\hbar}{2}\left[\begin{matrix}
      0 & 2 & 0 & 0 & 0\\
      -2 & 0 & \sqrt{6} & 0 & 0\\
      0 & -\sqrt{6} & 0 & \sqrt{6} & 0\\
      0 & 0 & -\sqrt{6} & 0 & 2\\
      0 & 0 & 0 & -2 & 0
    \end{matrix}\right] - \frac{i\hbar}{2}\left[\begin{matrix}
      0 & 2 & 0 & 0 & 0\\
      -2 & 0 & \sqrt{6} & 0 & 0\\
      0 & -\sqrt{6} & 0 & \sqrt{6} & 0\\
      0 & 0 & -\sqrt{6} & 0 & 2\\
      0 & 0 & 0 & -2 & 0
    \end{matrix}\right]\frac{\hbar}{2}\left[\begin{matrix}
      0 & 2 & 0 & 0 & 0\\
      2 & 0 & \sqrt{6} & 0 & 0\\
      0 & \sqrt{6} & 0 & \sqrt{6} & 0\\
      0 & 0 & \sqrt{6} & 0 & 2\\
      0 & 0 & 0 & 2 & 0
    \end{matrix}\right] =
    \\

    \\
    = \frac{i\hbar^2}{4}\left[\begin{matrix}
      -4 & 0 & 2\sqrt{6} & 0 & 0\\
      0 & -2 & 0 & 6 & 0\\
      -2\sqrt{6} & 0 & 0 & 0 & 2\sqrt{6}\\
      0 & -6 & 0 & 2 & 0\\
      0 & 0 & -2\sqrt{6} & 0 & 4
    \end{matrix}\right] -
    \frac{i\hbar^2}{4}\left[\begin{matrix}
      4 & 0 & 2\sqrt{6} & 0 & 0\\
      0 & 2 & 0 & 6 & 0\\
      -2\sqrt{6} & 0 & 0 & 0 & 2\sqrt{6}\\
      0 & -6 & 0 & -2 & 0\\
      0 & 0 & -2\sqrt{6} & 0 & -4
    \end{matrix}\right] =
    \\

    \\
    = i \hbar \hbar\left[\begin{matrix}
      -2 & 0 & 0 & 0 & 0\\
      0 & -1 & 0 & 0 & 0\\
      0 & 0 & 0 & 0 & 0\\
      0 & 0 & 0 & 1 & 0\\
      0 & 0 & 0 & 0 & 2
    \end{matrix}\right] = i \hbar L_3
  \end{array}
\end{equation}

We have proved everythin for all the values of j that we wanted. In the next section we will try to find the eigenvalues of this operators.

\pagelayout{margin} % margins

\section{Eigenvalues and unitary transformations}

During this chapter we are going to focus only in the case of $j=1$.

\marginnote[-1cm]{In most experiments we can only measure for j. If we consider and eigenvector of the form $e^{-i\frac{E}{\hbar}t}$, then if a state has an energy $E_1$, the state can not be a superoposition of states with different energies. However in our 3 dimensional model we know that the energy depends on j, but for j=1 we have 3 different states of the same energy. This means that this model allows degenerate states.}

To determine the eigen values of $L_1$ we need to get $\det(L_1-\lambda I) = 0$. We can do the same for $L_2$ and $L_3$. The eigenvalues of $L_1$ are:

\begin{equation}
  \begin{array}{c}
    \det(L_1-\lambda I) = 0 \Rightarrow
    \det\left[\begin{matrix}
      -\lambda & \frac{\hbar}{\sqrt{2}} & 0\\
      \frac{\hbar}{\sqrt{2}} & -\lambda & \frac{\hbar}{\sqrt{2}}\\
      0 & \frac{\hbar}{\sqrt{2}} & -\lambda
    \end{matrix}\right] = 0 \Rightarrow
    \\

    \\
    \Rightarrow \lambda\left(\lambda^2-\hbar^2\right) = 0 \Rightarrow
    \\

    \\
    \Rightarrow \lambda_1 = 0, \lambda_2 = \hbar, \lambda_3 = -\hbar
  \end{array}
\end{equation}

We can observed that the eigenvalues are the same as the ones of $L_3$. Because the conmmutation of the operators are closed under the equation \ref{9.6} we can rewrite our matrices using unitary transformation. We say that U is an unitary operator if:

\begin{equation}
  U^{\dagger}U = I
\end{equation}

We can rewrite now our matrices as:

\begin{equation}
  L_a' = U L_a U^{\dagger}
\end{equation}

And it can be proved that this new system commute under the same rule.

\begin{equation}
  \begin{array}{c}
    [L_a',L_b'] = UL_aU^{\dagger}UL_bU^{\dagger} - UL_bU^{\dagger}UL_aU^{\dagger} =
    \\

    \\
    = UL_aL_bU^{\dagger} - UL_bL_aU^{\dagger} = U[L_a,L_b]U^{\dagger} = i\hbar \epsilon_{abc} UL_cU^{\dagger} = i\hbar \epsilon_{abc} L_c'\\

    \\
    [L_a',L_b']= i\hbar \epsilon_{abc} L_c'
  \end{array}
\end{equation}

We want to find now the unitary matrix U that turns $L_1$ into $L_3$. As we saw in the beggining of the section $L_3$ is the diagonal matrix of $L_1$ so U will be the inverse of the matrix of eigenvalues.

\begin{equation}
  \begin{array}{c}
  L_1 = S L_3 S^{-1}
  \\

  \\
  L_3 = S^{-1} L_1 S
  \\

  \\
  L_1 = S^{\dagger} L_3 S \text{if ||\vec{v_{1,2,3}}||=1}
  \\

  \\
  L_3 = S L_1 S^{\dagger}
  \end{array}
\end{equation}

We can find the matrix S by finding the eigenvectors of $L_1$.

For $\lambda_1 = 0$

\begin{equation}
  \begin{array}{c}
    \left[\begin{matrix}
      0 & \frac{\hbar}{\sqrt{2}} & 0\\
      \frac{\hbar}{\sqrt{2}} & 0 & \frac{\hbar}{\sqrt{2}}\\
      0 & \frac{\hbar}{\sqrt{2}} & 0
    \end{matrix}\right]\left[\begin{matrix}
      x\\
      y\\
      z
    \end{matrix}\right] = \vec{0} \Rightarrow
    \\

    \\
    \Rightarrow y = 0, x = -z
    \\

    \\
    v_1 = \left( \begin{matrix}
      1\\
      0\\
      -1
    \end{matrix}\right)
  \end{array}
\end{equation}

For $\lambda_2 = \hbar$

\begin{equation}
  \begin{array}{c}
    \left[\begin{matrix}
      -\hbar & \frac{\hbar}{\sqrt{2}} & 0\\
      \frac{\hbar}{\sqrt{2}} & -\hbar & \frac{\hbar}{\sqrt{2}}\\
      0 & \frac{\hbar}{\sqrt{2}} & -\hbar
    \end{matrix}\right]
    \left[\begin{matrix}
      x\\
      y\\
      z
    \end{matrix}\right] = \vec{0} \Rightarrow
    \\

    \\
    \Rightarrow x = z, y = \sqrt{2}x
    \\

    \\
    v_2 = \left(\begin{matrix}
      1\\
      \sqrt{2}\\
      1
    \end{matrix}\right)
  \end{array}
\end{equation}

For $\lambda_3= -\hbar$

\begin{equation}
  \begin{array}{c}
    \left[\begin{matrix}
      \hbar & \frac{\hbar}{\sqrt{2}} & 0\\
      \frac{\hbar}{\sqrt{2}} & \hbar & \frac{\hbar}{\sqrt{2}}\\
      0 & \frac{\hbar}{\sqrt{2}} & \hbar
    \end{matrix}\right]
    \left[\begin{matrix}
      x\\
      y\\
      z
    \end{matrix}\right] = \vec{0} \Rightarrow
    \\

    \\
    \Rightarrow x = z, y = -\sqrt{2}x
    \\

    \\
    v_3 = \left(\begin{matrix}
      1\\
      -\sqrt{2}\\
      1
    \end{matrix}\right)
  \end{array}
\end{equation}

We can now build the matrix U.

\begin{equation}
  S=[c_3 v_3 | c_1 v_1 | c_2 v_2]
\end{equation}

\marginnote[-1cm]{We choose the order of the vectors in S to be the same as the order of the eigenvalues for $L_3$, to be the diagonal matrix we want to find.}

This matriz is not unique, there are multiple matrices that satisfied the equation, but if we want S to be unitary we need it to be formed by unitary vectors.

\begin{equation}
  S = \left[\begin{matrix}
    \frac{1}{2} & \frac{1}{\sqrt{2}} & \frac{1}{2}\\
    -\frac{\sqrt{2}}{2} & 0 & \frac{\sqrt{2}}{2}\\
    \frac{1}{2} & -\frac{1}{\sqrt{2}} & \frac{1}{2}
  \end{matrix}\right]
\end{equation}



We can check that this matrix is unitary by checking that $S^{\dagger}S = I$.

\begin{equation}
  S^{\dagger} S = \left[\begin{matrix}
    \frac{1}{2} & -\frac{\sqrt{2}}{2} & \frac{1}{2}\\
    \frac{1}{\sqrt{2}} & 0 & -\frac{1}{\sqrt{2}}\\
    \frac{1}{2} & \frac{\sqrt{2}}{2} & \frac{1}{2}
  \end{matrix}\right] \left[\begin{matrix}
    \frac{1}{2} & \frac{1}{\sqrt{2}} & \frac{1}{2}\\
    -\frac{\sqrt{2}}{2} & 0 & -\frac{\sqrt{2}}{2}\\
    \frac{1}{2} & -\frac{1}{\sqrt{2}} & \frac{1}{2}
  \end{matrix}\right] =
  \left[
    \begin{matrix}
      1 & 0 & 0\\
      0 & 1 & 0\\
      0 & 0 & 1
    \end{matrix}
  \right]
\end{equation}

Now we can say that $U = S^{\dagger}$ to get what we want. Now we can prove that $L_1' = L_3$.

\begin{equation}
  \begin{array}{c}
    L_1' = U L_1 U^{\dagger}
    \\

    \\
    L_1' = S^{\dagger} L_1 S = L_3
  \end{array}
\end{equation}

So we know that $L_1'$ is $L_3$. What about $L_2'$ and $L_3'$

\begin{equation}
  \begin{array}{c}
    L_2' = U L_2 U^{\dagger} =
    \\

    \\
    \left[\begin{matrix}
      1 & -\sqrt{2} & 1\\
      1 & 0 & 1\\
      1 & \sqrt{2} & 1
    \end{matrix}\right] \frac{i\hbar}{\sqrt{2}}\left[\begin{matrix}
      0 & 1 & 0\\
      -1 & 0 & 1\\
      0 & -1 & 0\\
    \end{matrix}\right]
    \left[
      \begin{matrix}
        1 & 1 & 1\\
        -\sqrt{2} & 0 & \sqrt{2}\\
        1 & 1 & 1
      \end{matrix}\right] =



  \end{array}
\end{equation}




\section{Spherical Coordinates}

We can describe the position of a point in space using spherical coordinates. We can define a point in space using the distance from the origin, the angle between the x axis and the projection of the point in the x-y plane and the angle between the z axis and the point. We can define the position of a point in space using the following equations:

\begin{equation}
  \begin{array}{c}
    x = r\sin(\theta)\cos(\phi)\\
    y = r\sin(\theta)\sin(\phi)\\
    z = r\cos(\theta)
  \end{array}
\end{equation}

\begin{marginfigure}[-4cm]
  \begin{tikzpicture}

  \draw[-] (-2,0) -- (2,0); % x-y axis
  \draw[-] (0,-2) -- (0,2); % z axis
  \filldraw[black] (2,0) circle (0pt) node [anchor=north]{$x-y$};
  \filldraw[black] (0,2) circle (0pt) node [anchor=east]{$z$};

  %vector
  \draw[->] (0,0) -- (1.5,1.5);
  \filldraw[black] (1.5,1.5) circle (0pt) node [anchor=south]{$\vec{r}$};
  %angle
  \draw[blue] (0,0.7) arc (45:0:0.7);
  \node at (0.2,0.8) {$\theta$};

  \end{tikzpicture}
  \caption{Spherical coordinates}
  \labfig{sph_coor1}
\end{marginfigure}

\begin{marginfigure}[1cm]
  \begin{tikzpicture}

  \draw[-] (-2,0) -- (2,0); % x axis
  \draw[-] (0,-2) -- (0,2); % y axis
  \filldraw[black] (2,0) circle (0pt) node [anchor=north]{$x$};
  \filldraw[black] (0,2) circle (0pt) node [anchor=east]{$y$};

  %vector
  \draw[->] (0,0) -- (1.5,-1.5);
  \filldraw[black] (1.5,-1.5) circle (0pt) node [anchor=north]{$\vec{r}\sin{\theta}$};

  %angle
  \draw[blue] (0.7,0) arc (0:-45:0.7);
  \node at (1.2,-0.3) {$\phi$};

  \end{tikzpicture}
  \caption{Spherical coordinates}
  \labfig{sph_coor2}
\end{marginfigure}


We can find the derivatives with:

\begin{equation}
  \begin{array}{c}
    \left(\begin{matrix}
      dx\\
      dy\\
      dz
    \end{matrix}\right) =
    \left(\begin{matrix}
      \frac{\partial x}{\partial r} & \frac{\partial x}{\partial \theta} & \frac{\partial x}{\partial \phi} \\
      \frac{\partial y}{\partial r} & \frac{\partial y}{\partial \theta} & \frac{\partial y}{\partial \phi} \\
      \frac{\partial z}{\partial r} & \frac{\partial z}{\partial \theta} & \frac{\partial z}{\partial \phi} \\
    \end{matrix}\right)
    \left(\begin{matrix}
      dr\\
      d\theta\\
      d\phi
    \end{matrix}
    \right) =
    \\

    \\
    = \left(\begin{matrix}
      \sin{\theta}\cos{\phi} & r\cos{\theta}\cos{\phi} & -r\sin{\theta}\sin{\phi}  \\
      \sin{\theta}\sin{\phi} & r\cos{\theta}\sin{\phi} & r\sin{\theta}\cos{\phi}   \\
      \cos{\theta}           & -r\sin{\theta}          &           0               \\
    \end{matrix}\right)
    \left(\begin{matrix}
      dr\\
      d\theta\\
      d\phi
    \end{matrix}
    \right)
  \end{array}
\end{equation}

\pagelayout{wide} % margins

We can get the expression for the derivatives of $r,\theta,\phi$ calculating the inverse of the matrix above.

\begin{equation}
  \begin{array}{c}
    \left(\begin{matrix}
      dr\\
      d\theta\\
      d\phi
    \end{matrix}
    \right) =
    \left(\begin{matrix}
      \sin{\theta}\cos{\phi} & \sin{\theta}\sin{\phi} & \cos{\theta} \\
      \frac{\cos{\theta}\cos{\phi}}{r} & \frac{\cos{\theta}\sin{\phi}}{r} & -\frac{\sin{\theta}}{r} \\
      -\frac{\sin{\phi}}{r\sin{\theta}} & \frac{\cos{\phi}}{r\sin{\theta}} & 0
    \end{matrix}\right)
    \left(\begin{matrix}
      dx\\
      dy\\
      dz
    \end{matrix}\right)
  \end{array}
\end{equation}

Mow we can say that:

\begin{equation}
  \begin{array}{c}
    \frac{\partial}{\partial x} = \cos{\phi}\sin{\theta}\frac{\partial}{\partial r} + \frac{\cos{\theta}\cos{\phi}}{r} \frac{\partial}{\partial \theta} - \frac{\sin{\phi}}{r\sin{\theta}} \frac{\partial}{\partial \phi}
    \\

    \\
    \frac{\partial}{\partial y} = \sin{\phi}\sin{\theta}\frac{\partial}{\partial r} + \frac{\cos{\theta}\sin{\phi}}{r} \frac{\partial}{\partial \theta} + \frac{\cos{\phi}}{r\sin{\theta}} \frac{\partial}{\partial \phi}
    \\

    \\
    \frac{\partial}{\partial z} = \cos{\theta}\frac{\partial}{\partial r} - \frac{\sin{\theta}}{r} \frac{\partial}{\partial \theta}
  \end{array}
\end{equation}

We can rewrite the angular momentum in spherical coordinates.

\begin{equation}
  \begin{array}{c}
    L_1 = -i\hbar\left[y\frac{\partial}{\partial z}-z\frac{\partial}{\partial y}\right] =
    \\

    \\
    -i\hbar\left((r\sin{\theta}\sin{\phi})\left[\cos{\theta}\frac{\partial}{\partial r} - \frac{\sin{\theta}}{r} \frac{\partial}{\partial \theta}\right]-r\cos{\theta}\left[\sin{\phi}\sin{\theta}\frac{\partial}{\partial r} + \frac{\cos{\theta}\sin{\phi}}{r} \frac{\partial}{\partial \theta} + \frac{\cos{\phi}}{r\sin{\theta}} \frac{\partial}{\partial \phi}\right]\right) =
    \\

    \\
    = -i\hbar\left(\left[-\sin^2\theta\sin\phi-\cos^2\theta\sin\phi\right]\frac{\partial}{\partial \theta}-\left[\cos\phi\frac{\cos\theta}{\sin\theta}\right]\frac{\partial}{\partial\phi}\right) = i\hbar\left[\sin\phi\frac{\partial}{\partial\theta}+\cos\phi\cot\theta\frac{\partial}{\partial\phi}\right]
    \\

    \\
    L_2 = -i\hbar\left[z\frac{\partial}{\partial x}-x\frac{\partial}{\partial z}\right] =
    \\

    \\
    = -i\hbar\left(\left[r\cos\theta\sin\theta\cos\phi-r\cos\theta\sin\theta\cos\phi\right]\frac{\partial}{\partial r} + \left[\cos^2\theta\cos\phi+\sin^2\theta\cos\phi\right]\frac{\partial}{\partial \theta}+\left[-\sin\phi\cot\theta\right]\frac{\partial}{\partial\phi}\right) =
    \\

    \\
    = i\hbar\left[-\cos\phi\frac{\partial}{\partial\theta}+\sin\phi\cot\theta\frac{\partial}{\partial\phi}\right]
    \\

    \\
    L_3 = -i\hbar\left[x\frac{\partial}{\partial y}-y\frac{\partial}{\partial x}\right] =
    \\

    \\
    = -i\hbar\left(\left[r\sin^2\theta\sin\phi\cos\phi-r\sin^2\theta\sin\phi\cos\phi\right]\frac{\partial}{\partial r} + \left[\cos\phi\cos\theta\sin\phi\sin\theta-\cos\phi\cos\theta\sin\phi\sin\theta\right]\frac{\partial}{\partial\phi}\right) =
    \\

    \\
    = -i\hbar\frac{\partial}{\partial \phi}
  \end{array}
\end{equation}

We can see that any of the components of the angular momentm depend on the radial coordinate or in the derivate of this one. Because of this we can think that the angular momentum is going to commute with a radial change.

\begin{equation}
  \begin{array}{c}
    [f(r) \text{ or } \frac{\partial}{\partial r},L_a] = 0
  \end{array}
\end{equation}

We are going to need the expression for $L_+$ and $L_-$.

\begin{equation}
  \begin{array}{c}
    L_+ = L_1 + iL_2 = \hbar e^{i\phi}\left[\frac{\partial}{\partial\theta}+i\cot\theta\frac{\partial}{\partial\phi}\right]
    \\

    \\
    L_- = L_1 - iL_2 = \hbar e^{-i\phi}\left[\frac{\partial}{\partial\theta}-i\cot\theta\frac{\partial}{\partial\phi}\right]
  \end{array}
\end{equation}

We are also interested in the operator L^2, as it is one of the operators than commutes and we are going to calculate it in 3 dimensions to get the wave equation in a future section. First we need the square of the 3 components of the angular momentum.

\begin{equation}
  \begin{array}{c}
    L_1^2 = -\hbar^2\left(y\frac{\partial}{\partial z}-z\frac{\partial}{\partial y}\right)\left(y\frac{\partial}{\partial z}-z\frac{\partial}{\partial y}\right)= -\hbar^2\left(y^2\frac{\partial^2}{\partial z^2}-y\frac{\partial}{\partial y}-2yz\frac{\partial^2}{\partial y\partial z}-z\frac{\partial}{\partial z}+z^2 \frac{\partial^2}{\partial y^2}\right)
    \\

    \\
    L_2^2 = -\hbar^2\left(z\frac{\partial}{\partial x}-x\frac{\partial}{\partial z}\right)\left(z\frac{\partial}{\partial x}-x\frac{\partial}{\partial z}\right) = -\hbar^2\left(z^2\frac{\partial^2}{\partial x^2}-z\frac{\partial}{\partial z}-2xz\frac{\partial^2}{\partial x\partial z}-x\frac{\partial}{\partial x}+x^2 \frac{\partial^2}{\partial z^2}\right)
    \\

    \\
    L_3^2 = -\hbar^2\left(x\frac{\partial}{\partial y}-y\frac{\partial}{\partial x}\right)\left(x\frac{\partial}{\partial y}-y\frac{\partial}{\partial x}\right) = -\hbar^2\left(x^2\frac{\partial^2}{\partial y^2}-x\frac{\partial}{\partial x}-2xy\frac{\partial^2}{\partial x\partial y}-y\frac{\partial}{\partial y}+y^2 \frac{\partial^2}{\partial x^2}\right)
    \\

    \\
    L^2 = L_1^2 + L_2^2 + L_3^2 =
    \\

    \\
    = -\hbar^2\left[\left(x^2+y^2+z^2\right)\left(\frac{\partial^2}{\partial x^2}+\frac{\partial^2}{\partial y^2}+\frac{\partial^2}{\partial z^2}\right)-x^2\frac{\partial^2}{\partial x^2}-y^2\frac{\partial^2}{\partial y^2}-z^2\frac{\partial^2}{\partial z^2}-2xy\frac{\partial^2}{\partial x\partial y}-2zx\frac{\partial^2}{\partial z\partial x}-2yz\frac{\partial^2}{\partial z\partial y}-2\left(x\frac{\partial}{\partial x}+y\frac{\partial}{\partial y}+z\frac{\partial}{\partial z}\right)\right] =
    \\

    \\
    = -\hbar^2\left[\left(x^2+y^2+z^2\right)\left(\frac{\partial^2}{\partial x^2}+\frac{\partial^2}{\partial y^2}+\frac{\partial^2}{\partial z^2}\right) - \left(x\frac{\partial}{\partial x}+y\frac{\partial}{\partial y}+z\frac{\partial}{\partial z}\right)^2-\left(x\frac{\partial}{\partial x}+y\frac{\partial}{\partial y}+z\frac{\partial}{\partial z}\right)\right]
  \end{array}
\end{equation}

Now that we have the expression for $L^2$ in cartesian coordinates, we can change it to spherical coordinates. We are going to solve by parts.

\begin{equation}
  \begin{array}{c}
    \left(x\frac{\partial}{\partial x}+y\frac{\partial}{\partial y}+z\frac{\partial}{\partial z}\right) = \left[r\sin^2\theta\cos^2\phi+r\sin^2\theta\sin^2\phi+r\cos^2\theta\right]\frac{\partial}{\partial r} + \left[\sin\theta\cos\phi\cos^2\phi+\sin\theta\cos\theta\sin^2\phi-\sin\theta\cos\theta\right]\frac{\partial}{\partial\theta}+\left[-\cos\phi\sin\phi+\sin\phi\cos\phi\right]\frac{\partial}{\partial \phi} =
    \\

    \\
    = r\frac{\partial}{\partial r}
  \end{array}
\end{equation}

With this relation we can finally get L^2.

\begin{equation}
  \begin{array}{c}
    L^2 = -h^2\left[r^2\left(\frac{\partial^2}{\partial x^2}\right)-\left(r\frac{\partial}{\partial r}\right)^2-r\frac{\partial}{\partial r}\right]
  \end{array}
\end{equation}


\pagelayout{margin} % margins

\section{The eigenfunction Y}

We've been labeling the eigen functions by $\ket{j,m}$, but we are going to rename it with:

\begin{equation}
  \begin{array}{c}
    Y_{j,m}(\theta,\phi) <=> \bra{\theta,\phi}\ket{j,m}
  \end{array}
\end{equation}

$Y_{j,m}$ is only a function of $\theta$ and $\phi$, because as we saw in the previous section the angular momentum only depends on these parameters. Let's see what the components of the angular momentum say about our function.

\begin{equation}
  \begin{array}{c}
    L_3 Y_{j,m} = -i\hbar\frac{\partial}{\partial \phi}Y_{j,m} = m\hbar Y_{j,m}
  \end{array}
\end{equation}

If we solved the differential equation we have that:

\begin{equation}
  \begin{array}{c}
    Y_{j,m} = e^{im\phi}P_j^m(\theta)
  \end{array}
\end{equation}

We haven't assume separation of variable it came out from the definition of the angular momentum on the z direction. If we look at the periodicity of the function:

\begin{equation}
  \begin{array}{c}
    Y_{j,m}(\theta,\phi+2\pi) = Y_{j,m}(\theta,\phi)
    \\

    \\
    e^{im(\phi+2\pi)} = e^{im\phi}
  \end{array}
\end{equation}

This implies that m has to be an integer number, wich implies that the half values are not allowed. Our j is gonna be restricted to only positive integers. We will have to wait for Dirac to know more about this.

We can find a recursive equation for $P_j^m(\theta)$ using $L_+$ and $L_-$.

\begin{equation}
  \begin{array}{c}
    L_+ Y_{j,m} = \hbar \sqrt{j(j+1)-m(m+1)}Y_{j,m+1} =
    \\

    \\
    \hbar e^{i\phi}\left[\frac{\partial}{\partial \theta}+i\cot\theta\frac{\partial}{\partial \phi}\right]P_j^m(\theta)e^{im\phi} = \hbar\sqrt{j(j+1)-m(m+1)}P_j^{m+1}(\theta)e^{i(m+1)\phi}
    \\

    \\
    \hbar e^{i(m+1)\phi} \left[\frac{\partial}{\partial \theta}P_j^m(\theta) -m\cot{\theta}P_j^m(\theta)\right] = \hbar\sqrt{j(j+1)-m(m+1)}P_j^{m+1}(\theta)e^{i(m+1)\phi}
    \\

    \\
    \left(\frac{\partial}{\partial\theta}-m\cot\theta\right)P_j^m(\theta) = \sqrt{j(j+1)-m(m+1)}P_j^{m+1}(\theta)
  \end{array}
\end{equation}

We also have a expresion that with the same logic come from $L_-$.

\begin{equation}
  \begin{array}{c}
    L_- Y_{j,m} = \hbar \sqrt{j(j+1)-m(m-1)}Y_{j,m+1} =
    \\

    \\
    \hbar e^{-i\phi}\left[\frac{\partial}{\partial \theta}-i\cot\theta\frac{\partial}{\partial \phi}\right]P_j^m(\theta)e^{im\phi} = \hbar\sqrt{j(j+1)-m(m-1)}P_j^{m-1}(\theta)e^{i(m-1)\phi}
    \\

    \\
    \left[\frac{\partial}{\partial \theta}P_j^m(\theta) +m\cot{\theta}P_j^m(\theta)\right] = -\sqrt{j(j+1)-m(m+1)}P_j^{m-1}(\theta)
    \\

    \\
    \left(\frac{\partial}{\partial\theta}+m\cot\theta\right)P_j^m(\theta) = -\sqrt{j(j+1)-m(m+1)}P_j^{m-1}(\theta)
  \end{array}
\end{equation}

For the case m=j in L_+ and m=-j in L_-.

\begin{equation}
  \begin{array}{c}
    \frac{\partial}{\partial\theta} P_j^j(\theta) - j\cot\theta P_j^j(\theta) = 0
    \\

    \\
    \frac{\partial}{\partial\theta} P_j^{-j}(\theta) - j\cot\theta P_j^{-j}(\theta) = 0
  \end{array}
\end{equation}

We can see that is the same equation which implies that $P_j^j = P_j^{-j}$. Now we are going to find the solution for it.

\begin{equation}
  \begin{array}{c}
    P_j^j(\theta) = N_j (\sin\theta)^j
  \end{array}
\end{equation}

To get the value of $N_j$ we can use the normalization condition:

\begin{equation}
  \begin{array}{c}
    \int_{0}^{2\pi}d\theta\int_{0}^{\pi}d\phi[Y_{j,m}(\theta,\phi)]^*Y_{j,m}(\theta,\phi)\sin\theta = 1
  \end{array}
\end{equation}

The most important result of this chapter is that the function is a product function of the two parameters becuase we have two operators than commute, ($L^2$ and $L_3$).

We can find this functions for every j, we will do it for j=1.

%

% INCLUDE THE SOLUTION FOR J=1

%


\section{The wave equation in 3 dimensions}

We are going to start multiplying the square of the angular momentun by $\frac{1}{2mr^2}$ to get something similar to the angular kinetic energy.

\begin{equation}
  \begin{array}{c}
    \frac{L^2}{2mr^2} = \frac{-\hbar^2}{2mr^2}\left[r^2\left(\frac{\partial^2}{\partial x^2}\right)-\left(r\frac{\partial}{\partial r}\right)^2-r\frac{\partial}{\partial r}\right]
  \end{array}
\end{equation}

We can get from the kinetic energy from the equation above.

\begin{equation}
  \begin{array}{c}
    \frac{-\hbar^2}{2m} \left(\frac{\partial^2}{\partial x^2}+\frac{\partial^2}{\partial y^2}+\frac{\partial^2}{\partial z^2}\right) = \frac{L^2}{2mr^2}-\frac{\hbar^2}{2mr^2}\left[\left(r\frac{\partial}{\partial r}\right)^2+r\frac{\partial}{\partial r}\right]
  \end{array}
\end{equation}

If we include the potential energy we will get the total energy operator $H$.

\begin{equation}
  \begin{array}{c}
    \frac{L^2}{2mr^2}-\frac{\hbar^2}{2mr^2}\left[\left(r\frac{\partial}{\partial r}\right)^2+r\frac{\partial}{\partial r}\right] + V(r) = H
  \end{array}
\end{equation}

We inmideatly can say that $H$ and $L^2$ commute because $L^2$ commutes with the angular kinetic energy the radial kinetic energy and the potential energy. It also going to commute with all the components of the angular momentum for the same reason.

\begin{equation}
  \begin{array}{c}
    [H,L^2] = 0
    \\

    \\
    [H,L_a] = 0
  \end{array}
\end{equation}

For any radial potential we have:

\begin{equation}
  \begin{array}{c}
    L_z \ket{E,j,m} = \hbar m \ket{E,j,m}
    \\

    \\
    L^2 \ket{E,j,m} = \hbar^2 j(j+1) \ket{E,j,m}
    \\

    \\
    H \ket{E,j,m} = E \ket{E,j,m}
  \end{array}
\end{equation}

The wave function is going to be:

\begin{equation}
  \begin{array}{c}
    \left[\frac{-\hbar^2}{2mr^2}\left[\left(r\frac{\partial}{\partial r}\right)^2+\left(r\frac{\partial}{\partial r}\right) \right]+V(r)+\frac{\hbar^2 j(j+1)}{2mr^2}\right] \ket{E,j,m} = \ket{E,j,m}
  \end{array}
\end{equation}

We are going to label the eigenfunction by $\psi_{E,j,m}(r,\theta,\phi)$. As we proved before because $H$,$L^2$ and $L_3$ we can say that the function is product function.

\begin{equation}
  \psi_{E,j,m}(r,\theta,\phi) = R_{E,j}(r) Y_{j,m}(\theta,\phi)
\end{equation}

The general wave equation becomes:

\begin{equation}
  \left{-\frac{\hbar^2}{2mr^2}\left[\left(r\frac{d}{dr}\right)^2+\left(r\frac{d}{dr}\right)-j(j+1)\right]+V(r)- E \right} R_{E,j} (r) = 0
\end{equation}

To solve the equation for every potential we are going to use some algebra tricks.

\begin{equation}
  \begin{array}{c}
    R_{E,j} (r) = \frac{U_{E,j}(r)}{r}
    \\

    \\
    \left(r\frac{d}{dr}\right)\frac{U}{r} = \frac{dU}{dr}-\frac{U}{r}
    \\

    \\
    \left(r\frac{d}{dr}\right)^2\frac{U}{r} = r\frac{d^2U}{dr^2}-\frac{dU}{dr}+\frac{U}{r}
    \\

    \\
    \left[\left(r\frac{d}{dr}\right)^2+r\frac{d}{dr}\right] \frac{U}{r} = r\frac{d^2U}{dr^2}
  \end{array}
\end{equation}

We can rewrite the wave equation as:

\begin{equation}
  \begin{array}{c}
    -\frac{\hbar^2}{2mr^2}\left[r\frac{d^2U}{dr^2}-j(j+1)\frac{U}{r}\right]+V(r)\frac{U}{r}-E\frac{U}{r} = 0
    \\

    \\
    \frac{d^2U_{E,j}(r)}{dr^2}+\frac{2m}{\hbar^2}\left(E-V(r)-\frac{\hbar^2}{2mr^2}j(j+1)\right)U_{E,j}(r) = 0
  \end{array}
\end{equation}

We know that the integral from 0 to infinite of U has to be finite which implies that the limit of U going to infinite has to be 0.

\begin{equation}
  \begin{array}{c}
    \lim_{r\to\infty} U_{E,j}(r) = 0
  \end{array}
\end{equation}

For r close to 0 we can assume that U is a polynomial function of r.

\begin{equation}
  \lim_{r\to 0} U_{E,j}(r) \to r^p
\end{equation}

The equation for this assumption is:

\begin{equation}
  p(p-1)r^{p-2} - j(j+1) r^{p-2} + \frac{2m}{\hbar^2}(E-V(r))r^p = 0
\end{equation}

If we assume that $V(r)r^p < r^{p-2}$ the last term goes to 0 faster than the rest of the terms when r goes to 0. \sidenote[-1cm]{This implies that $V(r)$ has to be weaker than 1/$r^2$}



\begin{equation}
  \begin{array}{c}
    p(p-1)r^{p-2} - j(j+1)r^{p-2} + 0 = 0
    \\

    \\
    p(p-1) = j(j+1)
    \\

    \\
    p_1 = j+1, p_2 = -j
  \end{array}
\end{equation}

If we solved the integral between 0 and a small value $\epsilon$ we get:

\begin{equation}
  \begin{array}{c}
    \int_{0}^{\epsilon} u^2 dr = \int_{0}^{\epsilon} r^{2p} dr = \frac{\epsilon^{2p+1}}{2p+1} \Rightarrow
    \\

    \\ \Rightarrow 2p+1 > 0 \Rightarrow p > -\frac{1}{2}
  \end{array}
\end{equation}

For the second solution of p we found that the only possible value is when $j=0$, i.e. $p=0$. We can also prove that j=0 is not a solution.

\begin{equation}
  \begin{array}{c}
    \text{if} p=0 \Rightarrow \psi = \frac{1}{r}P_j^m(\theta)e^{im\theta}
    \\

    \\
    K.E = \left(\vec{\nabla}\cdot\vec{\nabla}\right) \frac{f(\theta,\phi)}{r} = \delta(r)
  \end{array}
\end{equation}

Our potential comes from a central force so it can not have a delta function term, so the wave equation is going to be incompatible. This implies that p=-j=0 is not a solution.

We can say that the only possible value for p is j+1,for j=0,1,2,3...

